% !TEX encoding = UTF-8 Unicode
%%% This is a Markdown GitHub styled design template %%%

\documentclass[a4paper, 11pt]{article}

\usepackage[french]{babel}  % french language package
\frenchbsetup{StandardLists=true}  % english bullet lists
\usepackage{fontspec}
\usepackage[margin=2cm]{geometry}  % 2 centimetres margin

% Main font of the document:
\setmainfont{SF-Pro-Text}[Path = fonts/,
    					  UprightFont = *-Regular,
    					  BoldFont = *-Semibold,
					  	  ItalicFont = *-RegularItalic,
						  BoldItalicFont = *-SemiboldItalic]

% Mono font of the document:
\setmonofont{SFMono}[Path = fonts/,
					 UprightFont = *-Regular,
					 BoldFont = *-Bold,
					 ItalicFont = *-RegularItalic,
					 BoldItalicFont = *-BoldItalic,
					 Scale=0.9]

% Setting heading font for sections and subsections:
\newfontfamily\headingfont{SF-Pro-Display}[Path = fonts/,
										   UprightFont = *-Regular,
										   BoldFont = *-Semibold,
										   ItalicFont = *-RegularItalic,
										   BoldItalicFont = *-SemiboldItalic]

%% In case fontspec doesn't work with accents:
%\usepackage{polyglossia}
%\setmainlanguage{french}

\usepackage{url, hyperref}
\usepackage{verbatim}

\usepackage[table]{xcolor}

\usepackage[framemethod=TikZ]{mdframed}

\usepackage{tabularx}
\usepackage{array, multirow}
\usepackage{colortbl}  % table colours
\usepackage{tcolorbox}

\usepackage{listings}  % package listings
\usepackage{enumitem}

\usepackage[normalem]{ulem}  % scratched text

\usepackage{titlesec}

\newlength\tindent
\setlength{\tindent}{\parindent}
\setlength{\parindent}{0pt}
\renewcommand{\indent}{\hspace*{2.5em}}

\definecolor{darkgreen}{rgb}{0,0.6,0}
\definecolor{mauve}{rgb}{0.58,0,0.82}
\definecolor{darkred}{RGB}{167,29,93}
\definecolor{darkblue}{RGB}{24,54,145}

% New colours %
\definecolor{deep-gray}{RGB}{36,41,46}
\definecolor{darkgray}{RGB}{50,50,50}
\definecolor{gray}{RGB}{106,115,125}
\definecolor{table-gray}{RGB}{223,226,229}
\definecolor{light-gray}{RGB}{234,236,239}
\definecolor{code-gray}{RGB}{244,244,245}
\definecolor{block-gray}{RGB}{246,248,250}
\definecolor{red}{RGB}{215,58,73}
\definecolor{blue}{RGB}{0,92,197}
\definecolor{dark-blue}{RGB}{3,47,98}
\definecolor{purple}{RGB}{111,66,193}
\definecolor{orange}{RGB}{227,98,9}


% Old colours
\definecolor{cyan}{RGB}{93,217,239}
\definecolor{orange}{RGB}{253,151,31}
\definecolor{rose}{RGB}{249,38,114}
\definecolor{yellow}{RGB}{230,219,116}
\definecolor{green}{RGB}{82,190,91}
\definecolor{lime}{RGB}{166,226,46}
\definecolor{darkorange}{RGB}{233,83,37}

\color{deep-gray}  % general colour for the whole document

% Inline code frames:
\newcommand{\codex}[1]{\colorbox{code-gray}{\texttt{#1}}}  % inline code with sharp corners
\newcommand\code[2][fill=code-gray]{\tikz[baseline]\node[inner ysep=1pt, inner xsep=2.5pt, anchor=text, rectangle, rounded corners=2pt, #1]{\strut\texttt{#2}};}  % inline code with rounded corners
\newtcbox{\term}{on line, fontupper=\ttfamily\color{block-gray}, colback=deep-gray, boxrule=0pt, arc=2pt, boxsep=0pt, left=3pt, right=3pt, top=3pt, bottom=2.5pt} % inline terminal styled code

% Commands for syntax highlighting:
\newcommand{\key}[1]{\textcolor{red}{\texttt{#1}}}  % keyword
\newcommand{\func}[1]{\textcolor{purple}{\texttt{#1}}}  % function declaration
\newcommand{\param}[1]{\textcolor{deep-gray}{\texttt{\textsl{#1}}}}  % parameter
\newcommand{\built}[1]{\textcolor{blue}{\texttt{#1}}}  % built-in function
\newcommand{\op}[1]{\textcolor{red}{\texttt{#1}}}  % operator
\newcommand{\ent}[1]{\textcolor{blue}{\texttt{#1}}}  % number
\newcommand{\str}[1]{\textcolor{dark-blue}{\texttt{#1}}}  % string
\newcommand{\argm}[1]{\textcolor{orange}{\texttt{#1}}}  % keyword argument
\newcommand{\com}[1]{\textcolor{gray}{\texttt{#1}}}  % line comment

\mdfsetup{font=\ttfamily,
		  backgroundcolor=block-gray,
		  roundcorner=3pt,
		  hidealllines=true,
		  leftmargin=0,
		  rightmargin=0,
		  innerleftmargin=10,
		  innerrightmargin=30,
		  innertopmargin=10,
		  innerbottommargin=10}
		  

%----------------------------------------------------------------------------------------
%	CODE INCLUSION CONFIGURATION
%----------------------------------------------------------------------------------------

% -- Python configuration --
% Set the delimiter for classes declarations
\newcommand{\classClassHighlight}[1]{\textcolor{red}{class} \textcolor{purple}{#1}:}
% Set the delimiter for functions declarations
\newcommand{\functionDefHighlight}[1]{\textcolor{red}{def} \textcolor{purple}{#1}(}
% Set the delimiter for special methods declarations
\newcommand{\specialMethodHighlight}[1]{\textcolor{red}{def} \textcolor{blue}{\_\_#1\_\_}}

\lstdefinelanguage{Python} {
	% Standard keywords:
    morekeywords=[1]{and, as, assert, break, class, continue, def, del, elif, else, except,
    				 finally, for, from, global, if, import, in, is, lambda, nonlocal, not,
				 	 or, pass, raise, return, try, while, with, yield},
    %
    % Built-in functions:
    morekeywords=[2]{abs, all, any, ascii, bin, bool, bytearray, bytes, callable, chr,
    				 classmethod, compile, complex, delattr, dict, dir, divmod, enumerate,
				 	 eval, exec, False, filter, float, format, frozenset, getattr, globals,
					 hasattr, hash, help, hex, id, input, int, isinstance, issubclass,
					 iter, len, list, locals, map, max, memoryview, min, next, None,
					 object, oct, open, ord, pow, print, property, range, repr, reversed,
					 round, self, set, setattr, slice, sorted, staticmethod, str, sum,
					 super, True, tuple, type, vars, zip},
    %
    % Operators:
    % (The colour of otherkeywords is the same colour as keywordstyle[1])
    otherkeywords={+, -, *, /, =, <, >, \%, +=, -=, *=},
    %
    morecomment=[l][\color{gray}]{\#}, % Python comments (#)
    %
    morestring=[b]", % Strings defined with "
    morestring=[b]', % Strings defined with '
    %
    moredelim=[is][\classClassHighlight]{class\ }{:}, % Delimiter for classes declarations
	moredelim=[is][\functionDefHighlight]{def\ }{(}, % Delimiter for functions declarations
	moredelim=[is][\specialMethodHighlight]{def\ __}{__}, % Delimiter for special methods declarations
	moredelim=[s][\color{blue}]{__}{__}, % Delimiter for special methods
}


% -- Pseudo-code configuration --
% Set the delimiter for structure declarations
\newcommand{\structureHighlight}[1]{\textcolor{red}{structure} \textcolor{purple}{#1} \{}
% Set the delimiter for procedure (with arguments) declarations
\newcommand{\procedureHighlight}[1]{\textcolor{purple}{#1}(}
% Set the delimiter for functions declarations
\newcommand{\procedureWithoutArgumentsHighlight}[1]{\textcolor{red}{procédure} \textcolor{purple}{#1} \{}

\lstdefinelanguage{Pseudo-code} {
	% Mots-clés standards :
    morekeywords=[1]{alors, et, faire, non, ou, procédure, que, retourner, si, sinon,
    				 structure, tant},
    %
    % Fonctions intégrées :
    morekeywords=[2]{afficher, Faux, longueur, max, min, None, somme, Vrai},
    %
    % Operators:
    % (The colour of otherkeywords is the same colour as keywordstyle[1])
    otherkeywords={+, -, *, /, =, <, >, \%, +=, -=, *=},
    %
	morecomment=[l][\color{gray}]{//}, % Commentaires en ligne (//)
    morecomment=[l][\color{gray}]{\#}, % Commentaires python (#)
    %
    moredelim=[is][\structureHighlight]{structure\ }{\ \{}, % Délimiteur pour les déclarations de structure
	moredelim=[is][\procedureHighlight]{procedure\ }{(}, % Délimiteur pour les déclarations de procédure
	moredelim=[is][\procedureWithoutArgumentsHighlight]{procédure\ }{\ \{}, % Délimiteur pour les déclarations de procédure sans arguments
    morestring=[b]", % Chaînes de caractères définies avec : "
    morestring=[b]', % Chaînes de caractères définies avec : '
    %
}
                                          
% -- General configuration --
\lstset {
	frame=none, % No frame around code
	backgroundcolor=\color{block-gray}, % Background colour
	basicstyle=\linespread{1.1}\ttfamily, % Line spread and ttfamily font
	upquote=true, % Straight single quotes
	columns=flexible, % Flexible column alignment
	keepspaces=true, % Not drop spaces to fix column alignment and always convert tabulators to spaces
	commentstyle=\color{gray}, % Comments are gray
	stringstyle=\color{dark-blue}, % Strings are dark-blue
	showstringspaces=false, % Don't put marks in string spaces
	tabsize=4, % 4 spaces per tab
	%
	keywordstyle=[1]\color{red}, % Keywords style
	keywordstyle=[2]\color{blue}, % Built-in functions style
	%
	% Colours for numbers:
	% (The star indicates that literate replacements should not be made in strings, comments, and other delimited text)
	literate=*{0}{{{\color{blue}0}}}{1}
			  {1}{{{\color{blue}1}}}{1}
			  {2}{{{\color{blue}2}}}{1}
			  {3}{{{\color{blue}3}}}{1}
			  {4}{{{\color{blue}4}}}{1}
			  {5}{{{\color{blue}5}}}{1}
			  {6}{{{\color{blue}6}}}{1}
			  {7}{{{\color{blue}7}}}{1}
			  {8}{{{\color{blue}8}}}{1}
			  {9}{{{\color{blue}9}}}{1},
	%
	numbers=left, % Line numbers on left
	firstnumber=1, % Line numbers start with line 1
	numberstyle=\tt\scriptsize\color{gray}, % Line numbers are blue and small
	%stepnumber=5, % Line numbers go in steps of 5
}

\surroundwithmdframed[
  hidealllines=true, % Hide mdframed frame line border
  innerleftmargin=25pt, % Margin between mdframed's left border and lstlisting's numbers
  innertopmargin=3pt,
  innerbottommargin=3pt]{lstlisting}
  
% Creates a new command to include a python script:
% The first parameter is the filename of the script (without .py) and the second parameter is the caption
\newcommand{\pythonscript}[2]{
\begin{itemize}
\item[]\lstinputlisting[caption=#2, label=#1]{#1.py}
\end{itemize}
}

\titleformat{\section}{\huge\bfseries\headingfont}{\thesection}{1em}{}[{\color{light-gray}\titlerule[0.5pt]}]  % horizontal rule under sections
\titleformat{\subsection}{\LARGE\bfseries\headingfont}{\thesubsection}{1em}{}[{\color{light-gray}\titlerule[0.5pt]}]  % horizontal rule under subsections
\titleformat{\subsubsection}{\Large\bfseries\headingfont}{\thesubsection}{1em}{}[{\color{light-gray}}]
\titleformat*{\paragraph}{\large\bfseries\headingfont}
%\titleformat*{\subparagraph}{\large\bfseries}

\titlespacing*{\section}{0pt}{0.5cm}{0.5cm}  % 0.5cm above and below the section
\titlespacing*{\subsection}{0pt}{0.5cm}{0.5cm}  % 0.5cm above and below the subsection

\usepackage{setspace}  % package for line spacing
\onehalfspacing  % line spacing of one and a half


\begin{document}  % début du document
\pagestyle{plain}  % numéro en bas de page

\setcounter{secnumdepth}{0}  % pas de numérotation des sections

%\noindent
%Université de Caen Normandie \hfill Maël Querré\\
%Département d'informatique \hfill L2 Informatique, 2017--2018 \\

%\begin{center}
%	\Large{Sujet}\\
%	\vspace{0.5cm}
%\end{center}

\section{GitHub readme styled document}

Here is a sample {\bf Python} {\it listing}:

\begin{lstlisting}[language=Python]
def f(x):
	s = ("Test", 2 + 3, {'a': 'b'}, x)  # Comment
	if s:
		print(s[0].lower())

class Foo:
	def __init__(self):
		byte_string = 'newline:\n also newline:\x0a'
		text_string = "Cyrillic Я is \u042f."
		self.makeSense(whatever=1)
	
	def make_sense(self, whatever):
		self.sense = whatever

x = len('abc')
print(f.__doc__)
\end{lstlisting}

\subsection{Headers}

\begin{mdframed}
\# H1 \\
\#\# H2 \\
\#\#\# H3 \\
\#\#\#\# H4 \\
\#\#\#\#\# H5 \\
\#\#\#\#\#\# H6 \\

Alternatively, for H1 and H2, an underline-ish style: \\

Alt-H1 \\
====== \\

Alt-H2 \\
------
\end{mdframed}

\section{H1}

\subsection{H2}

\subsubsection{H3}

Alternatively, for H1 and H2, an underline-ish style:

\section{Alt-H1}

\subsection{Alt-H2}


\subsection{Emphasis}

\begin{mdframed}
Emphasis, aka italics, with *asterisks* or \_underscores\_. \\

Strong emphasis, aka bold, with **asterisks** or \_\_underscores\_\_. \\

Combined emphasis with **asterisks and \_underscores\_**. \\

Strikethrough uses two tildes. $\sim\sim$Scratch this.$\sim\sim$
\end{mdframed}

Emphasis, aka italics, with asterisks or underscores. \\

Strong emphasis, aka bold, with \textbf{asterisks} or \textbf{underscores}. \\

Combined emphasis with \textbf{asterisks and \textit{underscores}}. \\

Strikethrough uses two tildes. \sout{Scratch this.}


\subsection{Lists}

(In this example, leading and trailing spaces are shown with with dots: $\cdot$)

\begin{mdframed}
1. First ordered list item \\
2. Another item \\
$\cdot\;\cdot$* Unordered sub-list. \\
1. Actual numbers don't matter, just that it's a number \\
$\cdot\;\cdot$1. Ordered sub-list \\
4. And another item. \\

$\cdots$You can have properly indented paragraphs within list items. Notice the blank line above, and the leading spaces (at least one, but we'll use three here to also align the raw Markdown). \\

$\cdots$To have a line break without a paragraph, you will need to use two trailing spaces.$\cdot\;\cdot$ \\
$\cdots$Note that this line is separate, but within the same paragraph.$\cdot\;\cdot$ \\
$\cdots$(This is contrary to the typical GFM line break behaviour, where trailing spaces are not required.) \\

* Unordered list can use asterisks \\
- Or minuses \\
+ Or pluses
\end{mdframed}

\begin{enumerate}
	\item First ordered list item
	\item Another item
\end{enumerate}

\begin{itemize}[label=\textbullet]
	\item Unordered sub-list.
\end{itemize}

\begin{enumerate}
	\item Actual numbers don't matter, just that it's a number \\
	\item Ordered sub-list \\
	\item And another item. \\
	
	You can have properly indented paragraphs within list items. Notice the blank line above, and the leading spaces (at least one, but we'll use three here to also align the raw Markdown). \\
	
	To have a line break without a paragraph, you will need to use two trailing spaces.
Note that this line is separate, but within the same paragraph.
(This is contrary to the typical GFM line break behaviour, where trailing spaces are not required.)
\end{enumerate}

\begin{itemize}[label=\textbullet]
	\item Unordered list can use asterisks
	\item Or minuses
	\item Or pluses
\end{itemize}

\subsection{Links}

There are two ways to create links.


\vspace{2cm}

\subsection{Code and Syntax Highlighting}

Code blocks are part of the Markdown spec, but syntax highlighting isn't. However, many renderers -- like Github's and Markdown Here -- support syntax highlighting. Which languages are supported and how those language names should be written will vary from renderer to renderer. Markdown Here supports highlighting for dozens of languages (and not-really-languages, like diffs and HTTP headers); to see the complete list, and how to write the language names, see the highlight.js demo page.
es
\begin{mdframed}
Inline `code` has `back-ticks around` it.
\end{mdframed}

Inline \code{code} has \code{back-ticks around} it. \\

Blocks of code are either fenced by lines with three back-ticks \code{` ` `}, or are indented with four spaces. I recommend only using the fenced code blocks -- they're easier and only they support syntax highlighting.


\subsection{Tables}

Tables aren't part of the core Markdown spec, but they are part of GFM and \textit{Markdown Here} supports them. They are an easy way of adding tables to your email -{-} a task that would otherwise require copy-pasting from another application. \\

Colons can be used to align columns. \\

\rowcolors{2}{block-gray}{white}  % coloration du tableau
\renewcommand{\arraystretch}{2}  % espacement vertical à l'intérieur des cellules
\setlength{\tabcolsep}{0.5cm}  % espacement horizontal à l'intérieur des cellules
\begin{tabular}{|p{3cm}|p{3cm}|p{1cm}|}
\arrayrulecolor{table-gray}
\hline
\centering\textbf{Tables} & \centering\textbf{Are} & \centering\textbf{Cool} \tabularnewline
\hline
col 3 is & \centering right-aligned & \raggedleft\$1600 \tabularnewline
\hline
col 2 is & \centering centered & \raggedleft\$12 \tabularnewline
\hline
zebra stripes & \centering are neat & \raggedleft\$1 \tabularnewline
\hline
\end{tabular}

\vspace{0.5cm}

There must be at least 3 dashes separating each header cell. The outer pipes (|) are optional, and you don't need to make the raw Markdown line up prettily. You can also use inline Markdown. \\

\rowcolors{2}{block-gray}{white}  % coloration du tableau
\renewcommand{\arraystretch}{2}  % espacement vertical à l'intérieur des cellules
\setlength{\tabcolsep}{0.5cm}  % espacement horizontal à l'intérieur des cellules
\begin{tabular}{|p{2cm}|p{2cm}|p{1cm}|}
\arrayrulecolor{table-gray}
\hline
\centering\textbf{Markdown} & \centering\textbf{Less} & \centering\textbf{Pretty} \tabularnewline
\hline
\textit{Still} & \centering \code{renders} & \raggedleft \textbf{nicely} \tabularnewline
\hline
1 & 2 & 3 \tabularnewline
\hline
\end{tabular}


\subsection{Blockquotes}

\end{document}